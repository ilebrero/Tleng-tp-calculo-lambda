\documentclass{article}

\usepackage[utf8]{inputenc}

\usepackage{caratula}

\usepackage{graphicx}
\usepackage{enumerate}

\usepackage{amssymb}
\usepackage{mathtools}
\usepackage{amsmath}
\usepackage{amsthm}

\usepackage{algorithm}
\usepackage{algpseudocode}
\usepackage{listingsutf8}

\usepackage{float}
\floatplacement{figure}{h!}

\usepackage{geometry}
\usepackage{wrapfig}
\usepackage{cite}
\usepackage{dsfont}
\usepackage{xcolor}

\usepackage[space]{grffile}

\geometry{
 a4paper,
 total={210mm,297mm},
 left=30mm,
 right=30mm,
 top=30mm,
 bottom=30mm,
}
 
\usepackage{booktabs}

\usepackage{verbatim}
\usepackage{qtree}

\newtheorem{theorem}{Teorema}[section]
\newtheorem{corollary}{Corolario}[theorem]
\newtheorem{lemma}{Lema}[theorem]
 
\theoremstyle{definition}
\newtheorem{definition}{Definición}[section]
 
\theoremstyle{remark}
\newtheorem*{remark}{Observación}
 
\begin{document}
% Estos comandos deben ir antes del \maketitle
\materia{Teoria de Lenguajes} % obligatorio

\titulo{TP1}
\subtitulo{Analizador Sintáctico y Semántico para $\lambda^{bn}$ \\ \today}
\grupo{Grupo: }
 
\integrante{Francisco Leto}{}{}
\integrante{Ignacio Lebrero Rial}{}{}
\integrante{Federico Beuter}{827/13}{federicobeuter@gmail.com}
 
\maketitle

\tableofcontents

\pagebreak

\section{Introduccion}

En este infome se describe la implementacion de un analizador sintactico y semantico para $\lambda^{bn}$. En base al lenguaje presentado se procedio a construir una gramatica acorde, la cual luego fue utilizada para implementar el correspondiente Lexer y Parser. La implementacion fue realizado empleando Python junto con la libreria PLY (\textbf{P}ython \textbf{L}ex-\textbf{Y}acc) \footnote{PLY (Python Lex-Yacc): http://www.dabeaz.com/ply/ply.html}.

\section{Gramatica}

Debido a la naturaleza del calculo lambda tipado, la gramatica suele ser bastante simple, este caso no fue distinto. Lo mas importante a tener en cuenta es la asociatividad de las aplicaciones, en este caso es a la izquierda, es decir \texttt{M N P = (M N) P}. La gramatica libre de contexto resultante fue la siguiente:

\begin{verbatim}
AGREGAR LA GRAMATICA FINAL

Grammar

Rule 0     S' -> exp
Rule 1     exp -> L_BRACKET exp R_BRACKET
Rule 2     exp -> TRUE
Rule 3     exp -> FALSE
Rule 4     exp -> IF exp THEN exp ELSE exp
Rule 5     exp -> LAMBDA var POINT exp
Rule 6     exp -> exp exp
Rule 7     exp -> L_BRACKET exp R_BRACKET exp
Rule 8     exp -> ZERO
Rule 9     exp -> ISZERO L_BRACKET exp R_BRACKET
Rule 10    exp -> PRED L_BRACKET exp R_BRACKET
Rule 11    exp -> SUCC L_BRACKET exp R_BRACKET
Rule 12    var -> VAR_DECLARATION
Rule 13    exp -> VAR_USAGE
\end{verbatim}

Los simbolos terminales son:

\begin{verbatim}
AGREGAR SIMBOLOS TERMINALES

Terminals, with rules where they appear

ELSE                 : 4
FALSE                : 3
IF                   : 4
ISZERO               : 9
LAMBDA               : 5
LAMBDA_TYPE          : 
L_BRACKET            : 1 7 9 10 11
POINT                : 5
PRED                 : 10
R_BRACKET            : 1 7 9 10 11
SUCC                 : 11
THEN                 : 4
TRUE                 : 2
TYPE                 : 
VAR_DECLARATION      : 12
VAR_USAGE            : 13
ZERO                 : 8
error                : 
\end{verbatim}

Los simbolos no terminales son:

\begin{verbatim}
AGREGAR SIMBOLOS NO TERMINALES

exp                  : 1 4 4 4 5 6 6 7 7 9 10 11 0
var                  : 5
\end{verbatim}

\section{Lexer}

Para la identificacion de tokens empleamos las siguientes expresiones regulares:

\begin{center}
	\begin{tabular}{@{}ll@{}}
		\toprule
			Token & Expresion Regular \\ \midrule
			VAR\_DECLARATION &  {[}a-z\textbar A-Z{]}:((Nat\textbar Bool)-\textgreater (Nat\textbar Bool)\textbar Nat\textbar Bool)\\
			VAR\_USAGE &  {[}a-z\textbar A-Z{]}\\
			LAMBDA &  \textbackslash\\
			TYPE &  AGREGAR\\
			POINT & .\\
			L\_BRACKET & (\\
			R\_BRACKET & )\\
			ZERO & 0 \\
			LAMBDA\_TYPE & -\textgreater\\
		\bottomrule
	\end{tabular}
\end{center}

\section{Analisis Semantico}

Nuestro parser valida lo siguiente (AGREGAR LO QUE FALTE O ME HAYA COMIDO POR BOLUDO):

\begin{itemize}
	\item Se asegura de que la condicion del \texttt{if} sea del tipo booleano.
	\item Se asegura de que ambas opciones del \texttt{if} tengan el mismo tipo.
	\item Se asegura de que el tipo de la expresion lambda sea valido, es decir, \texttt{bool} o \texttt{nat}
	\item Se asegura de que en caso de que una expresion lambda sea evaluada, el valor con la que se la evalua se corresponda al propio de la expresion.
	\item Se asegura de que \texttt{succ} y \texttt{pred} se apliquen solamente a valores de tipo \texttt{nat}.
	\item Se asegura de que \texttt{isZero} unicamente se aplique a valores de tipo \texttt{bool}.
\end{itemize}

\section{Evaluaciones}

\subsection{Expresiones correctas}

\begin{tabular}{|l|l|}
\hline
Expresion                                                                    & Evaluacion                                          \\ \hline
if true then (\textbackslash x:Bool.false) else (\textbackslash x:Bool.true) & \textbackslash x:Bool.false:Bool-\textgreater Bool   \\ \hline
\textbackslash x:Nat.iszero(x)                                               & \textbackslash x:Nat.iszero(x):Nat-\textgreater Bool \\ \hline
pred(succ(succ(0)))                                                          & succ(0):Nat                                         \\ \hline
\end{tabular}

\subsection{Expresiones incorrectas}

\begin{tabular}{|l|l|}
\hline
Expresion                                            & Salida                                                   \\ \hline
if true then false else (\textbackslash x:Bool.true) & ERROR: Las dos opciones del if deben tener el mismo tipo \\ \hline
\textbackslash x.Not.succ(x)                         & Hubo un error en el parseo. Sintaxis invalida            \\ \hline
succ(iszero(true))                                   & ERROR: iszero espera un valor de tipo Nat                \\ \hline
\end{tabular}

\section{Manual de Usuario}

Para poder ejecutar el parser, es necesario emplear Python 3 con PLY instalado. Una vez que se tienen todas las dependencias, se puede ejecutar con cualquiera de los siguientes dos comandos:

\begin{itemize}
	\item \texttt{python CLambda.py}
	\item \texttt{python CLambda.py expresionLambda}
\end{itemize}

El segundo recibe la expresion lambda directamente y la evalua, mientras que el primero recibe la expresion por \texttt{stdin} mediante la consola. Los resultados de la evaluacion se imprimen por \texttt{stdout} en caso de que sea satisfactoria, en el caso contrario se imprime el motivo de error por \texttt{stderr}.

\section{Código fuente}

\subsection{CLambda.py}
\begin{small}
  \verbatiminput{../CLambda.py}
\end{small}

\subsection{parser\_processing.py}
\begin{small}
  \verbatiminput{../parser_processing.py}
\end{small}

\subsection{lambda\_lexer.py}
\begin{small}
  \verbatiminput{../lambda_lexer.py}
\end{small}

\subsection{lambda\_parser.py}
\begin{small}
  \verbatiminput{../lambda_parser.py}
\end{small}

\subsection{expressions.py}
\begin{small}
  \verbatiminput{../expressions.py}
\end{small}

\end{document}