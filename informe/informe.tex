\documentclass{article}

\usepackage[utf8]{inputenc}

\usepackage{caratula}

\usepackage{graphicx}
\usepackage{enumerate}

\usepackage{amssymb}
\usepackage{mathtools}
\usepackage{amsmath}
\usepackage{amsthm}

\usepackage{algorithm}
\usepackage{algpseudocode}
\usepackage{listingsutf8}
%\usepackage{listings}

\usepackage{float}
\floatplacement{figure}{h!}

\usepackage{geometry}
\usepackage{wrapfig}
\usepackage{cite}
\usepackage{dsfont}
\usepackage{xcolor}

\usepackage[space]{grffile}
 
\usepackage{booktabs}

\usepackage{verbatim}
\usepackage{qtree}

\newtheorem{theorem}{Teorema}[section]
\newtheorem{corollary}{Corolario}[theorem]
\newtheorem{lemma}{Lema}[theorem]
 
\theoremstyle{definition}
\newtheorem{definition}{Definición}[section]
 
\theoremstyle{remark}
\newtheorem*{remark}{Observación}
 
\lstset{
	basicstyle=\itshape,
	xleftmargin=3em,
	literate={->}{$\rightarrow$}{2}
} 
 
\begin{document}
% Estos comandos deben ir antes del \maketitle
\materia{Teoria de Lenguajes} % obligatorio

\titulo{TP1}
\subtitulo{Analizador Sintáctico y Semántico para $\lambda^{bn}$ \\ \today}
\grupo{Grupo: }
 
\integrante{Francisco Leto Mera}{249/09}{leto.francisco@gmail.com}
\integrante{Ignacio Lebrero Rial}{751/13}{ignaciolebrero@gmail.com}
\integrante{Federico Beuter}{827/13}{federicobeuter@gmail.com}
 
\maketitle

\tableofcontents

\pagebreak

\section{Introduccion}

En este infome se describe la implementacion de un analizador sintactico y semantico para $\lambda^{bn}$. En base al lenguaje presentado se procedio a construir una gramatica acorde, la cual luego fue utilizada para implementar el correspondiente Lexer y Parser. La implementacion fue realizado empleando Python junto con la libreria PLY (\textbf{P}ython \textbf{L}ex-\textbf{Y}acc) \footnote{PLY (Python Lex-Yacc): http://www.dabeaz.com/ply/ply.html}.

\section{Gramatica}

\subsection{Lexer}

El primer paso para la construcción de nuestra gramática fue la identificación de los \emph{tokens} y palabras reservadas. Utilizamos las siguientes palabras reservadas y literales: \textbf{if, then, else, succ, pred, iszero, true, false, `(', `)', `.', `\textbackslash', 0}. Definimos, ademas, los siguientes tokens, que se identificaron con las expresiones regulares detalladas a continuación.

\begin{center}
	\begin{tabular}{@{}ll@{}}
		\toprule
		Token & Expresion Regular \\ \midrule
		VAR\_DECLARATION &  {[}a-z\textbar A-Z{]+}:((Nat\textbar Bool)-\textgreater (Nat\textbar Bool)\textbar Nat\textbar Bool)\\
		VAR\_USAGE &  {[}a-z\textbar A-Z{]+}\\
		\bottomrule
	\end{tabular}
\end{center}

Ademas, el lexer ignora los caracteres en blanco y descompone el token de VAR\_DECLARATION en un objeto con el valor de la variable y su tipo. En la implementación, los literales se definieron como tokens para facilitar la implementación.

La gramática propuesta intenta ser lo mas simple posible, teniendo en cuenta la precedencia y asociatividad de los operadores, y evitando ambigüedades.


\begin{lstlisting}
E -> (E) | A
A -> \ D.E | C
D -> VAR_DECLARATION
C -> if E then E else E | P
P -> E E
F -> succ(E) | pred(E) | iszero(E) | G
G -> 0 | true | false | V
V -> VAR_USAGE
\end{lstlisting}

Hay un claro problema de ambigüedad en $ P \rightarrow E E$ que resolvimos en la implementación determinando a mano la precedencia. En total PLY presenta $20$ conflictos, que desembocan principalmente de este, pero junto con lo definido a mano, y el default de PLY en caso de conflicto (hacer \emph{shift}) es suficiente para implementar el lenguaje deseado adecuadamente.

\section{Analisis Semantico}

Nuestro codigo, utilizando \emph{PLY} construye un parser LARL para la gramática propuesta. Luego le agregamos semantica, que permite realizar el chequeo e inferencia de tipo, junto con la evaluación de las cadenas bien formadas. Parte de lo implementado consiste en: 

\begin{itemize}
	\item Asegurar que la guarda de la condición sea del tipo booleano.
	\item Asegurar que ambas ramas de la condición tengan el mismo tipo.
	\item Se asegura de que el tipo de la variable en la abstracción lambda sea valido.
	\item Se asegura de que \texttt{succ} y \texttt{pred} y \texttt{iszero} se apliquen solamente a valores de tipo \texttt{Nat}.
	\item Infiere los tipos cuando esto es posible.
	\item Respeta la precedencia y asociatividad al evaluar la cadenas de entrada.
\end{itemize}

\section{Evaluaciones}

\subsection{Expresiones correctas}

\begin{tabular}{|l|l|}
\hline
Expresion                                                                    & Evaluacion                                          \\ \hline
if true then (\textbackslash x:Bool.false) else (\textbackslash x:Bool.true) & \textbackslash x:Bool.false:Bool-\textgreater Bool   \\ \hline
\textbackslash x:Nat.iszero(x)                                               & \textbackslash x:Nat.iszero(x):Nat-\textgreater Bool \\ \hline
pred(succ(succ(0)))                                                          & succ(0):Nat                                         \\ \hline
\end{tabular}

\subsection{Expresiones incorrectas}

\begin{tabular}{|l|l|}
\hline
Expresion                                            & Salida                                                   \\ \hline
if true then false else (\textbackslash x:Bool.true) & ERROR: Las dos opciones del if deben tener el mismo tipo \\ \hline
\textbackslash x.Not.succ(x)                         & Hubo un error en el parseo. Sintaxis invalida            \\ \hline
succ(iszero(true))                                   & ERROR: iszero espera un valor de tipo Nat                \\ \hline
\end{tabular}

\section{Manual de Usuario}

Para poder ejecutar el parser, es necesario emplear Python 3 con PLY instalado. Es posible instalar las dependencias ejecutando \textbf{pip install -r requirements.txt }Una vez que se tienen todas las dependencias, se puede ejecutar con cualquiera de los siguientes dos comandos:

\begin{itemize}
	\item \texttt{python CLambda.py}
	\item \texttt{python CLambda.py expresionLambda}
\end{itemize}

El segundo recibe la expresión lambda directamente y la evalua, mientras que el primero recibe la expresion por \texttt{stdin} mediante la consola. Los resultados de la evaluación se imprimen por \texttt{stdout} en caso de que sea satisfactoria, en el caso contrario se imprime el motivo de error por \texttt{stderr} y se para la ejecución con código de salida $1$.

En caso de utilizar la primera opción, es importante tener en cuenta el correcto escapado de los caracteres en la terminal. Para evitar estos problemas, se recomienda la segunda opción.

Ademas, se brinda el archivo \emph{tests.py} que utilizando la libreria de \emph{Python} de \emph{Unit Test} valida $13$ casos de prueba distintos, que figuran en el enunciado y sirven como validación extra respecto a los ejemplos brindados previamente. 

\section{Detalles de implementación}

El código fuente utilizado puede encontrarse en el capitulo \ref{codigo} pero a continuación aclaramos algunas de las decisiones de diseño que consideramos más importantes:

\begin{itemize}
	\item Utilizamos los tipos nativos de \emph{Python} para la implementación de la semantica para facilitar las operaciones, y representamos los tipos de lambda como una tupla.
	
	\item Todos los métodos de las expresiones utilizadas implementan el método \emph{\_\_str\_\_} propio de \emph{Python que permite representarlos como un string}. Utilizando esto podemos convertir cada expresión en un string de salida cuando es necesario (cuando forma parte de un valor lambda, por ejemplo) y cada método tiene las particularidades del mismo a la hora de procesarlo.
	
	\item Al utilizar los typos de \emph{Python} los $Nat$ son convertidos a números durante la evaluación. Al finalizar la misma son reconvertidos a una anidación de \emph{succ} utilizando una expresión regular.
	
	\item Utilizamos el marcado especial \emph{APP} para indicar la asociación de la aplicación en las reglas sintácticas.
	
	\item La función \emph{get\_type\_str} se encarga de convertir los tipos internos de \emph{Python} en su correcta representación en Cálculo Lambda cuando esto es necesario.
	
\section{Concluciones}

	Disfrutamos mucho de la realización del TP y encontramos muy grato implementar lo que se estuvo viendo a lo largo del cuatrimestre. PLY como una implementación pura en \emph{Python} no es muy performante, pero brinda excelentes herramientas para generar parsers \emph{LALR} y es sencillo de usar, aunque su uso intensivo de meta-programación requiere un conocimiento bastante avanzado del lenguaje.
	
	En general consideramos que realizamos un trabajo satisfactorio. Fuimos capaces de construir una gramática para el lenguaje y brindar un parser LALR que la parsee y evalué, realizando el chequeo e inferencia de tipos en el proceso.
\end{itemize}

\newpage
\section{Código fuente}\label{codigo}
\lstset{backgroundcolor=\color{white}} 
\subsection{CLambda.py}
\begin{small}
  \verbatiminput{../Clambda.py}
\end{small}

\subsection{parser\_processing.py}
\begin{small}
  \verbatiminput{../parser_processing.py}
\end{small}

\subsection{lambda\_lexer.py}
\begin{small}
  \verbatiminput{../lambda_lexer.py}
\end{small}

\subsection{lambda\_parser.py}
\begin{small}
  \verbatiminput{../lambda_parser.py}
\end{small}

\subsection{expressions.py}
\begin{small}
  \verbatiminput{../expressions.py}
\end{small}

\end{document}